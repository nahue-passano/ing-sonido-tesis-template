En el Marco Teórico debemos incorporar la bibliografía, artículos de revistas, ponencias de congresos, links de Internet o todo aquello que haya contribuido a formar el cuerpo del saber sobre el que va a basarse la investigación, incorporando los procesos y ecuaciones necesarios.  
Puede ser uno varios capítulos donde se detallen los parámetros, indicadores y conceptos teóricos referentes al tema a tratar. Se recomienda no utilizar conceptos muy básicos, como definición de nivel de presión sonora, ponderación A, etc.

\textbf{Estado del arte}

Puede ser uno o varios capítulos que desarrollen el estado del arte del área de conocimiento donde se inserta la tesis. La profundidad del enfoque en el tratamiento de los temas debe ser adecuado para el entendimiento posterior de los resultados y discusiones de la tesis. No es necesario que sea autocontenido, es recomendable el uso amplio de referencias a trabajos previos que se encuentren en la literatura abierta sobre el tema.