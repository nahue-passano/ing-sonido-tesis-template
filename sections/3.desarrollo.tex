El desarrollo tecnológico propuesto se estructura en diversas fases y etapas, según se detalla en los objetivos específicos de la investigación. 

\subsection{Recolección y conformación del conjunto de datos}

En la etapa inicial, es necesario realizar una investigación exhaustiva y meticulosa de los conjuntos de datos disponibles. Mas específicamente, son de principal interés los conjuntos de datos que tengan discursos en el dialecto rioplatense del idioma español.
A su vez
discursos de figuras emblemáticas de la cultura argentina. Dicha fase incluye la selección de los oradores y de sus discursos, elementos cruciales para la emulación de sus voces. Para asegurar segmentos de habla de alta calidad y de extensa duración, se propone recolectar material audiovisual de podcasts o entrevistas, de los cuales se extraen segmentos específicos del hablante, verificando que estén libres de ruidos externos o efectos de postproducción como música o sonidos indeseados.
Tras completar la etapa inicial de composición de los datos crudos, es necesario conformar una base de datos con un formato específico por orador para facilitar el entrenamiento del sintetizador. Para ello, se propone el desarrollo de una plataforma auxiliar que automatice el preprocesamiento de los audios y genere la base de datos requerida. Esta plataforma debe contener un sistema de reconocimiento automático de habla, o ASR (por sus siglas en inglés, Automatic Speech Recognition), capaz de transcribir los discursos a texto y segmentarlos en porciones de diversas duraciones. Asimismo, la plataforma debe organizar tanto los audios como las transcripciones en el formato exigido por el modelo. Para el reconocimiento de habla, se aconseja emplear Whisper (Radford et al., 2023), un sistema destacado por su rapidez y eficacia, representando el estado del arte en tecnologías de ASR.
Dado que las transcripciones se realizan mediante un modelo de reconocimiento automático de habla, pueden presentar errores debido a la calidad del audio o a imprecisiones inherentes al sistema de ASR. Para mitigar este problema, se sugiere la implementación de una plataforma auxiliar que permita la validación de las transcripciones automatizadas. Dicha plataforma debe mostrar la transcripción junto con el discurso correspondiente, posibilitando su escucha y lectura en simultáneo. Tras la reproducción de un discurso, se determina si la transcripción es acertada o no, lo cual facilita la prevención de incluir transcripciones erróneas dentro de la base de datos necesaria.
Ambas plataformas auxiliares están construidas en Python utilizando la librería Streamlit (Streamlit, 2018), la cual facilita la creación de aplicaciones web de manera sencilla y con mínimo esfuerzo. Estas plataformas se alojan en HuggingFace (\cite{huggingface}), una plataforma reconocida en el campo de la inteligencia artificial por su amplia colección de modelos de vanguardia y ofrecer acceso para desplegar aplicaciones web.

\subsubsection{Conjuntos de datos disponibles}
Para la presente investigación se decide tomar en cuenta solo los conjuntos de datos que tengan discursos en español. Entre los mas destacados, se encuentra el conjunto de Mozilla llamado Common Voice (REFERENCIA), y el conjunto llamado Multilingual LibriSpeech (https://arxiv.org/abs/2012.03411) (REFERENCIA)

Hablar de Librispeech crowdsourced etc

\subsubsection{Recopilación de discursos con hablantes argentinos}
hablar de sacar charlas de youtube

\subsubsection{Conformación del conjunto de datos}
Hablar de las plataformas desarrolladas

\subsubsection{Análisis exploratorio del conjunto de datos conformado}

EDA con

- cantidad de hablantes, por género y por pais.
- cantidad de horas, por género y por pais.
- distribución del pitch, por género y por país.

\subsubsection{Estructuración jerárquica de los discursos y transcripciones}
TODO
- Formatos de dataset
- Formato elegido

\subsection{Entrenamiento y ajuste del sintetizador}

Una vez conformada la base de datos necesaria para entrenar el modelo, se debe configurar adecuadamente el entorno donde se ejecutará el entrenamiento. En este trabajo se utilizan servicios de cómputo en la nube, como los ofrece la plataforma AWS (\cite{amazonwebservices}), ya que ofrece un amplio espectro de opciones respecto a capacidades y costos. Con el entorno debidamente configurado, se procede a entrenar el sistema.
Como se comenta anteriormente, en esta etapa se pone especial énfasis en el entrenamiento del sintetizador, ya que es el responsable de aprender y emplear las características propias del dialecto rioplatense del español. Durante la fase de entrenamiento, el modelo procesa lotes de audios junto con sus transcripciones correspondientes para generar un espectrograma de Mel que refleje fielmente el discurso original. En este proceso, se calcula el espectrograma de Mel del discurso original para tenerlo como referencia a la hora de evaluar el generado por el sintetizador. Esta evaluación se realiza mediante una función de costo o de pérdida, la cual indica cuán alejado está el espectrograma sintético del objetivo. Con el valor de la distancia ya calculada, se modifican los pesos internos del modelo de forma tal que en la próxima iteración la distancia disminuya, minimizando el error que produce el sintetizador. Este proceso de optimización se repite hasta obtener un sintetizador capaz de generar espectrogramas muy similares a los originales. El entrenamiento mencionado se debe realizar para cada hablante por separado, es decir, primero se entrena el sintetizador con una base de datos amplia del dialecto rioplatense del idioma español, y luego se procede a realizar un ajuste fino (o también llamado fine-tuning) con los datos de cada hablante.

\subsection{Diseño de la plataforma y despliegue del sistema.}

Con el sistema ya entrenado, se desarrolla la plataforma con la cual los usuarios pueden interactuar con el modelo previsto. Dicha plataforma contará con diversas funcionalidades:

\begin{itemize}
    \item Ingreso del libro o texto: Los usuarios tienen a disposición un cuadro de carga para subir el libro deseado en formato pdf, o también un cuadro para ingresar directamente el texto que quieren convertir a discurso.
    \item Selección del orador: Se provee una lista de oradores disponibles, entre los cuales se encuentran celebridades reconocidas de la cultura argentina.
    \item Configuración de la generación: El usuario puede ajustar variables del discurso como pitch y velocidad de habla, lo cual otorga flexibilidad a la hora de generar discursos personalizados.
    \item Visualización y escucha del discurso: Se muestra en pantalla un objeto para reproducir el discurso o bien descargar el mismo en formato de alta calidad.

\end{itemize}



\subsection{Stack tecnológico}