Una vez recolectada la información de la prueba subjetiva, se procede a realizar un análisis de correlación entre las variables medidas (calidad, naturalidad y similitud) con los parámetros objetivos elegidos. También se busca comparar los resultados objetivos y subjetivos con implementaciones del estado del arte para determinar la competitividad y relevancia del sistema dentro del ámbito actual.