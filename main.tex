\documentclass[12pt]{article}
\usepackage[a4paper]{geometry}
\usepackage[spanish,es-tabla]{babel}
\usepackage[utf8]{inputenc}
\usepackage[T1]{fontenc}
% \usepackage{fontspec}
\usepackage{graphicx}
\usepackage{times}
\usepackage{tipa}
\usepackage{amsmath}
\usepackage{setspace}
\usepackage{enumitem}
\usepackage{fancyhdr}
\usepackage{lipsum}
\usepackage{parskip}
\usepackage[backend=biber, style=apa]{biblatex}
\addbibresource{bibliografia.bib}
\usepackage{caption}
\usepackage{floatrow}
\usepackage{titlesec}
\usepackage{titletoc}
\usepackage{appendix}
\usepackage{tabularray}
\usepackage{tocloft} 
\usepackage{physics}
\usepackage{changepage}
\usepackage[table]{xcolor}
\usepackage[hidelinks]{hyperref} % eliminar rectangulos rojos en los links

\usepackage{titlesec}
\titleformat{\section}
  {\normalfont\fontsize{16}{16}\bfseries}{\thesection .}{0.5ex}{}
\titleformat{\subsection}
  {\normalfont\fontsize{14}{14}\bfseries}{\thesubsection .}{0.5ex}{}
\titleformat{\subsubsection}
  {\normalfont\fontsize{14}{14}\bfseries}{\thesubsubsection.}{0.5ex}{}
\titleformat{\paragraph}
{\normalfont\normalsize\bfseries}{\theparagraph}{1em}{}
\titlespacing*{\paragraph}
{0pt}{3.25ex plus 1ex minus .2ex}{1.5ex plus .2ex}

%-----------------CONFIGURACION-------------------------------

% Configuracion de la Pagina 
\geometry{
    a4paper,
    total={21cm,29.7cm},
    left=3cm,
    top=2.5cm,
    right=2.5cm,
    bottom=2.5cm }

\setstretch{1.5} %Interlineado a 1.5
\setlength{\parindent}{1cm} %Sangría de cada nuevo párrafo a 1cm

% Configuración del pie de página
\fancyfoot[C]{\rule{\textwidth}{0.4pt}\\ \hspace*{\fill}\fontsize{12pt}{14pt}\selectfont Desarrollo de un sistema automático de audiolibros con prosodia rioplatense. \hspace*{\fill}\thepage} 

%\setlength{\footskip}{1.02 cm}
\setlength{\footskip}{0.5 cm}
\pagestyle{fancy} % Establecer el estilo de página como 'fancy'

\fancyhead{} % Limpiar encabezado
\renewcommand{\headrulewidth}{0pt} % Eliminar la línea de encabezado

\DefineBibliographyStrings{spanish}{ % Redefinir la cadena de texto para Bibliografía
  references = {BIBLIOGRAFÍA}, % <-- Aquí cerramos correctamente la definición}

% Configuración global de las descripciones de las figuras
\captionsetup[figure]{labelsep=period}
% Configuración global de las descripciones de las tablas
\floatsetup[table]{capposition=top}
\captionsetup[table]{labelsep=period}

% Personalización del formato del índice de figuras
\renewcommand{\cftfigpresnum}{Figura }
\renewcommand{\cftfigaftersnum}{.}
\setlength{\cftfignumwidth}{4em}

% Personalización del formato del índice de tablas
\renewcommand{\cfttabpresnum}{Tabla }
\renewcommand{\cfttabaftersnum}{.}
\setlength{\cfttabnumwidth}{4em}

% Personalizar el formato del índice general
\renewcommand{\cftsecleader}{\cftdotfill{\cftdotsep}} % Agrega puntos entre el título de la sección y el número de página
\renewcommand{\cftsecfont}{\bfseries} % Establece la fuente del título de la sección en el índice como negrita
\renewcommand{\cftsecpagefont}{\bfseries} % Establece la fuente de la página de la sección en el índice como negrita


% Definir el espaciado entre sections y subsections
\titlespacing*{\subsection}{1cm}{*1.5}{*1}

% Definir el espaciado entre subsections y subsubsections
\titlespacing*{\subsubsection}{1.5cm}{*1.5}{*1}

\usepackage{fontspec}
\setromanfont[
BoldFont=Calibri Bold.TTF,
ItalicFont=Calibri Italic.ttf,
BoldItalicFont=Calibri Bold Italic.ttf,
]{Calibri Regular.ttf}

\begin{document}

%-------------------------- PORTADA -----------------
\begin{titlepage}
\centering
\includegraphics[width=13.58cm, height=3.1cm]{Logo Untref.png} 

\vspace{0.1cm}

\hspace*{-1.31cm}% Espacio horizontal de 1.31cm desde el borde izquierdo de la hoja
\begin{minipage}[t]{16cm}
\centering
\framebox[17.13cm][c]{\parbox{18.13cm}{\centering{\fontsize{16pt}{1.5pt}\selectfont INGENIERÍA DE SONIDO}}}
\vspace{0.5cm} % Espacio vertical entre el recuadro y el texto siguiente

\end{minipage}


\vspace{36pt}

{\bfseries\fontsize{22pt}{18pt} \selectfont Títutlo de la tesis \par}

\vspace{42pt}

% {\itshape\fontsize{18pt}{24pt}\selectfont \textbf{Subtítulo de la tesis (si lo tuviera)} \par}

\vspace{64pt}

{\centering\itshape\fontsize{14pt}{1pt}\selectfont Tesis final presentada para obtener el título de Ingeniero\par}
{\centering\itshape\fontsize{14pt}{1pt}\selectfont de Sonido de la Universidad Nacional de Tres de Febrero \par}
{\centering\itshape\fontsize{14pt}{1pt}\selectfont (UNTREF) \par}

\vspace{70pt}

{\bfseries\fontsize{14pt}{0pt}\selectfont TESISTA: Nombre y Apellido (DNI número) \par}
{\bfseries\fontsize{14pt}{0pt}\selectfont TUTOR: Nombre y Apellido (Ing., PhD, etc.) \par}
{\bfseries\fontsize{14pt}{0pt}\selectfont COTUTOR: Nombre y Apellido (Ing., PhD, etc.) \par}

\vfill

\begin{table}[h]
\hrulefill \\ 
\begin{tabular}{c}
Fecha de defensa: mes y año $\lvert$ Locación: Buenos Aires, Argentina \\
\end{tabular}
\end{table}
\hrule
\end{titlepage}

\newpage
\thispagestyle{empty} % Página en blanco sin número de página, encabezado ni pie de página
\mbox{} % Contenido de la página en blanco
\newpage

%---------------------------------------------------------------
% CUERPO DEL DOCUMENTO
% Centrar título de sección

\pagenumbering{roman} % Comienza la numeración de página en números romanos
\setcounter{page}{2} % Establece el número de página según sea necesario
\setcounter{secnumdepth}{4}
%-----------------Agradecimientos-----------------
\newpage 
\thispagestyle{plain} 
% \section*{AGRADECIMIENTOS} % Título centrado
\begin{centering}
\Large{\textbf{AGRADECIMIENTOS}}

\end{centering}

Se propone incluir este apartado, donde se debe agradecer primeramente a las autoridades de la Universidad, al coordinador de la carrera, al tutor y a los docentes implicados en el desarrollo de la investigación. Seguidamente agradecer a familiares o a aquellas personas que se quiera. También puede incluirse en la siguiente hoja una dedicatoria personal. A modo de ejemplo el contenido podría ser:

“En primer lugar dar gracias a la Universidad Nacional de Tres de Febrero (UNTREF), a su Rector Emérito Lic. Anibal Jozami y Rector Lic. Martín Kaufmann, a todo su personal docente y no docente. Por promover un espacio ideal para el desarrollo de ideas y nuevos pensamientos y brindar a todos y cada uno de los alumnos, de esta casa de altos estudios, todos los recursos que esta institución dispone.   
Esta investigación no hubiera sido posible sin una formación académica acorde, por este motivo debo extender mi agradecimiento a los docentes de la carrera de Ingeniería de Sonido de la UNTREF, a su coordinador Ing. Alejandro Bibondo, que siendo la primera carrera de estas características del país, es muy importante contar con un cuerpo docente afín a las exigencias que este desafío propone, prestando su dedicación y vocación de enseñar. 
Un especial agradecimiento por la participación de esta tesis a la tutora Ing. Nombre Apellido, que supo transmitirme sus conocimientos y ayudarme a organizarme y fijarme un rumbo concreto y delineado, disponiendo desmedidamente de su tiempo. 
Por otra parte, quisiera hacer una mención especial al Ing. Hernan San Martin, que permitió el uso de las instalaciones de su laboratorio para poder trabajar y la disposición de todos sus recursos para que dicha investigación se realizara en tiempo y forma. 
Por último y no menos importante, quiero dar un afectuoso y cálido agradecimiento a mi familia…”


\newpage


\clearpage

\newpage
\thispagestyle{plain} % Página en blanco sin número de página, encabezado ni pie de página
\mbox{} % Contenido de la página en blanco
\newpage

%-----------------Estado del Arte-----------------
\newpage
\thispagestyle{plain} 
% \section*{DEDICATORIA} 
\textcolor{white}{
hola \\
hola \\
hola \\
hola \\
}
\begin{flushright}
\LARGE{\textbf{DEDICATORIA}}
\end{flushright}


\begin{flushright}
\textit{Elige a quién o a qué quieres dedicárselo \\
Elegir el motivo de la dedicatoria (orientativo)
}
\end{flushright}


%------------------Indices------------------------
\newpage
\renewcommand{\contentsname}{ÍNDICE DE CONTENIDOS}
\tableofcontents   % Índice General
\listoffigures  % Índice de Figuras
\listoftables   % Índice de Tablas 


%------------------Resumen------------------------

\newpage
\thispagestyle{plain} 
% \section*{\centering RESUMEN} 
\addcontentsline{toc}{section}{RESUMEN} % Agregar RESUMEN al índice
\begin{centering}
\Large{\textbf{RESUMEN}}

\end{centering}

% \raggedright
Su contenido no debe superar una página. Se indicarán los objetivos del trabajo, los métodos y resultados principales. A dos espacios debajo del resumen, en la misma página, se colocarán hasta 5 palabras clave que identifican los contenidos del trabajo.



\textbf{Palabras Clave:} 

% \end{titlepage}

%------------------Abstrac------------------------

\newpage
\thispagestyle{plain} 
% \section*{\centering ABSTRACT} 
\addcontentsline{toc}{section}{ABSTRACT} % Agregar ABSTRACT al índice
\begin{centering}
\Large{\textbf{ABSTRACT}}

\end{centering}

Ídem que para castellano. \\

\textbf{Keywords:} 




%-----------------Introducción-----------------
\clearpage % Asegúrate de que la sección de introducción comience en una nueva página
\pagenumbering{arabic} % Restaurar numeración arábiga
\section{INTRODUCCIÓN} 
\subsection{Fundamentación}

Breve fundamentación del trabajo de tesis en forma clara y concisa, estableciendo la relevancia y pertinencia del mismo en el ámbito del conocimiento donde se desarrolle. Debería establecerse la ubicación del trabajo dentro del estado del arte, pero sin desarrollar el detalle que se hará en los siguientes capítulos. No deben incluirse resultados ni detalles experimentales. Es deseable que no supere las 4 hojas.


\subsection{Objetivos}

\subsubsection{Objetivo general}

Resumir en una sola frase el objetivo de la investigación.

\subsubsection{Objetivos específicos}

\begin{itemize}
    \item Listar los distintos objetivos siguiendo un orden metodológico. Recordar que la revisión bibliográfica no es un objetivo específico.
    \item Identificar los recintos, paramentos verticales y horizontales y revestimientos fonoabsorbentes que conforman el estudio de grabación.
    \item Realizar simulaciones de tiempo de reverberación e indicadores de inteligibilidad de la palabra mediante el programa EASE.
    \item Obtener el índice de reducción sonora R de los distintos detalles constructivos mediante el programa de predicción INSUL.
    \item Realizar mediciones in situ de los descriptores de acondicionamiento acústico según norma ISO 3382-1.
    \item Realizar mediciones in situ según norma ISO 16283-1 para obtener los índices de aislamiento acústico a ruido aéreo.
    \item Comparar los resultados de las simulaciones respecto a los valores obtenidos de las mediciones.
    \item Realizar propuestas de mejora a implementar.
\end{itemize}


\subsection{Estructura de la investigación}

Indicar en distintos párrafos el contenido de cada uno de los capítulos que conformarán la tesis.
En el capítulo 2 se presenta el marco teórico donde se detallan los parámetros e indicadores necesarios para el desarrollo de la investigación.
En el capítulo 3 se hace referencia a aquellas investigaciones vinculadas con el tema a desarrollar en esta tesis.
En el capítulo 4…




\clearpage



%-----------------Marco Teorico-----------------
\newpage
\section{MARCO TEÓRICO Y ESTADO DEL ARTE} 
    En el Marco Teórico debemos incorporar la bibliografía, artículos de revistas, ponencias de congresos, links de Internet o todo aquello que haya contribuido a formar el cuerpo del saber sobre el que va a basarse la investigación, incorporando los procesos y ecuaciones necesarios.  
Puede ser uno varios capítulos donde se detallen los parámetros, indicadores y conceptos teóricos referentes al tema a tratar. Se recomienda no utilizar conceptos muy básicos, como definición de nivel de presión sonora, ponderación A, etc.

\textbf{Estado del arte}

Puede ser uno o varios capítulos que desarrollen el estado del arte del área de conocimiento donde se inserta la tesis. La profundidad del enfoque en el tratamiento de los temas debe ser adecuado para el entendimiento posterior de los resultados y discusiones de la tesis. No es necesario que sea autocontenido, es recomendable el uso amplio de referencias a trabajos previos que se encuentren en la literatura abierta sobre el tema.

% %-----------------Estado del Arte-----------------
% \newpage
% \section{ESTADO DEL ARTE} 
%     \input{sections/Estado del Arte}

%-----------------Desarrollo de la herramienta-----------------
\newpage
\section{DESARROLLO DE LA HERRAMIENTA} 
    El desarrollo tecnológico propuesto se estructura en diversas fases y etapas, según se detalla en los objetivos específicos de la investigación. 

\subsection{Recolección y conformación del conjunto de datos}

En la etapa inicial, es necesario realizar una investigación exhaustiva y meticulosa de los conjuntos de datos disponibles. Mas específicamente, son de principal interés los conjuntos de datos que tengan discursos en el dialecto rioplatense del idioma español.
A su vez
discursos de figuras emblemáticas de la cultura argentina. Dicha fase incluye la selección de los oradores y de sus discursos, elementos cruciales para la emulación de sus voces. Para asegurar segmentos de habla de alta calidad y de extensa duración, se propone recolectar material audiovisual de podcasts o entrevistas, de los cuales se extraen segmentos específicos del hablante, verificando que estén libres de ruidos externos o efectos de postproducción como música o sonidos indeseados.
Tras completar la etapa inicial de composición de los datos crudos, es necesario conformar una base de datos con un formato específico por orador para facilitar el entrenamiento del sintetizador. Para ello, se propone el desarrollo de una plataforma auxiliar que automatice el preprocesamiento de los audios y genere la base de datos requerida. Esta plataforma debe contener un sistema de reconocimiento automático de habla, o ASR (por sus siglas en inglés, Automatic Speech Recognition), capaz de transcribir los discursos a texto y segmentarlos en porciones de diversas duraciones. Asimismo, la plataforma debe organizar tanto los audios como las transcripciones en el formato exigido por el modelo. Para el reconocimiento de habla, se aconseja emplear Whisper (Radford et al., 2023), un sistema destacado por su rapidez y eficacia, representando el estado del arte en tecnologías de ASR.
Dado que las transcripciones se realizan mediante un modelo de reconocimiento automático de habla, pueden presentar errores debido a la calidad del audio o a imprecisiones inherentes al sistema de ASR. Para mitigar este problema, se sugiere la implementación de una plataforma auxiliar que permita la validación de las transcripciones automatizadas. Dicha plataforma debe mostrar la transcripción junto con el discurso correspondiente, posibilitando su escucha y lectura en simultáneo. Tras la reproducción de un discurso, se determina si la transcripción es acertada o no, lo cual facilita la prevención de incluir transcripciones erróneas dentro de la base de datos necesaria.
Ambas plataformas auxiliares están construidas en Python utilizando la librería Streamlit (Streamlit, 2018), la cual facilita la creación de aplicaciones web de manera sencilla y con mínimo esfuerzo. Estas plataformas se alojan en HuggingFace (\cite{huggingface}), una plataforma reconocida en el campo de la inteligencia artificial por su amplia colección de modelos de vanguardia y ofrecer acceso para desplegar aplicaciones web.

\subsubsection{Conjuntos de datos disponibles}
Para la presente investigación se decide tomar en cuenta solo los conjuntos de datos que tengan discursos en español. Entre los mas destacados, se encuentra el conjunto de Mozilla llamado Common Voice (REFERENCIA), y el conjunto llamado Multilingual LibriSpeech (https://arxiv.org/abs/2012.03411) (REFERENCIA)

Hablar de Librispeech crowdsourced etc

\subsubsection{Recopilación de discursos con hablantes argentinos}
hablar de sacar charlas de youtube

\subsubsection{Conformación del conjunto de datos}
Hablar de las plataformas desarrolladas

\subsubsection{Análisis exploratorio del conjunto de datos conformado}

EDA con

- cantidad de hablantes, por género y por pais.
- cantidad de horas, por género y por pais.
- distribución del pitch, por género y por país.

\subsubsection{Estructuración jerárquica de los discursos y transcripciones}
TODO
- Formatos de dataset
- Formato elegido

\subsection{Entrenamiento y ajuste del sintetizador}

Una vez conformada la base de datos necesaria para entrenar el modelo, se debe configurar adecuadamente el entorno donde se ejecutará el entrenamiento. En este trabajo se utilizan servicios de cómputo en la nube, como los ofrece la plataforma AWS (\cite{amazonwebservices}), ya que ofrece un amplio espectro de opciones respecto a capacidades y costos. Con el entorno debidamente configurado, se procede a entrenar el sistema.
Como se comenta anteriormente, en esta etapa se pone especial énfasis en el entrenamiento del sintetizador, ya que es el responsable de aprender y emplear las características propias del dialecto rioplatense del español. Durante la fase de entrenamiento, el modelo procesa lotes de audios junto con sus transcripciones correspondientes para generar un espectrograma de Mel que refleje fielmente el discurso original. En este proceso, se calcula el espectrograma de Mel del discurso original para tenerlo como referencia a la hora de evaluar el generado por el sintetizador. Esta evaluación se realiza mediante una función de costo o de pérdida, la cual indica cuán alejado está el espectrograma sintético del objetivo. Con el valor de la distancia ya calculada, se modifican los pesos internos del modelo de forma tal que en la próxima iteración la distancia disminuya, minimizando el error que produce el sintetizador. Este proceso de optimización se repite hasta obtener un sintetizador capaz de generar espectrogramas muy similares a los originales. El entrenamiento mencionado se debe realizar para cada hablante por separado, es decir, primero se entrena el sintetizador con una base de datos amplia del dialecto rioplatense del idioma español, y luego se procede a realizar un ajuste fino (o también llamado fine-tuning) con los datos de cada hablante.

\subsection{Diseño de la plataforma y despliegue del sistema.}

Con el sistema ya entrenado, se desarrolla la plataforma con la cual los usuarios pueden interactuar con el modelo previsto. Dicha plataforma contará con diversas funcionalidades:

\begin{itemize}
    \item Ingreso del libro o texto: Los usuarios tienen a disposición un cuadro de carga para subir el libro deseado en formato pdf, o también un cuadro para ingresar directamente el texto que quieren convertir a discurso.
    \item Selección del orador: Se provee una lista de oradores disponibles, entre los cuales se encuentran celebridades reconocidas de la cultura argentina.
    \item Configuración de la generación: El usuario puede ajustar variables del discurso como pitch y velocidad de habla, lo cual otorga flexibilidad a la hora de generar discursos personalizados.
    \item Visualización y escucha del discurso: Se muestra en pantalla un objeto para reproducir el discurso o bien descargar el mismo en formato de alta calidad.

\end{itemize}



\subsection{Stack tecnológico}

%-----------------Evaluación de la herramienta-----------------
\newpage
\section{EVALUACIÓN DE LA HERRAMIENTA DISEÑADA} 
    En esta sección, se establecen las variables necesarias para la evaluación objetiva y subjetiva del sistema. En particular, se enfoca en medir el desempeño y valorar la percepción de los discursos generados por el sistema.

\subsection{Evaluación objetiva}



\subsection{Evaluación subjetiva}

Para realizar la evaluación subjetiva se plantea realizar una encuesta del tipo Mean Opinion Score (MOS) (ITU-T). El MOS es un procedimiento que permite obtener percepciones subjetivas sobre la calidad (MOS), similitud (Sim-MOS) y naturalidad (Nat-MOS) de las voces sintéticas generadas. Para llevar a cabo esta evaluación, se seleccionan diez muestras de audio generadas por el sistema, correspondientes a diez oradores diferentes, y se presentan a un grupo de oyentes. Se procura que la composición del grupo mantenga un equilibrio entre oyentes expertos en tópicos de audio y oyentes generales. A fin de que la muestra de participantes sea significativa y representativa, se necesita contar con al menos veinticinco participantes. Cada participante escucha estas muestras y se solicita calificarlas en una escala del 1 al 5 en función de tres criterios clave: calidad, similitud y naturalidad. La calidad se refiere a cuán bien se percibe la fidelidad y la claridad del audio sintetizado. La similitud evalúa cuán cercana es la voz sintética a la voz original del hablante de referencia. La naturalidad se refiere a cuán fluido y humano suena el discurso sintetizado. Es importante destacar que el orden de presentación de las muestras se debe organizar de manera aleatoria para cada participante en pos de evitar sesgos potenciales. Además, cada muestra de audio se debe escuchar una sola vez, garantizando que las calificaciones se basan en las impresiones inmediatas de los participantes. Por último se realiza una pregunta sobre la experiencia del participante en escucha crítica y/o uso de sistemas de TTS.


%-----------------Validación-----------------
% \newpage
% \section{VALIDACIÓN DE LA EVALUACIÓN SUBJETIVA} 
%     La validación de las pruebas comienza una vez recolectados los datos de la encuesta subjetiva. Para cada respuesta, se calcula el MOS, el Nat-MOS y el Sim-MOS promediando las respuestas para cada par de audios presentados. Luego se descartan las respuestas anómalas (también llamadas outliers) para poder realizar un análisis de normalidad y homocedasticidad. Dichas pruebas se realizan en Python con la librería statsmodels (Seabold \& Perktold, 2010).

%-----------------Análisis de resultados-----------------
\newpage
\section{ANÁLISIS DE RESULTADOS} 
    Aquí se aplicarán los métodos estadísticos que nos permitan concluir los resultados y proyectar diferentes facetas de los datos obtenidos. Como las pruebas no se realizarán todavía, deben plantearse los métodos elegidos y explicarlos.

%-----------------Conclusiones-----------------
\newpage
\section{CONCLUSIONES} 
    El desarrollo propuesto resulta ser una herramienta de gran utilidad para producir audiolibros de calidad, personalizables y a gran velocidad. Los resultados objetivos y subjetivos expuestos muestran que el sistema es capaz de narrar textos de alta calidad con gran similitud a la voz objetivo propuesta, superando en rendimiento a implementaciones anteriores en el dialecto rioplatense del Español.

%-----------------Linea futuras de Invetigacion-----------------
\newpage
\section{LÍNEAS FUTURAS DE INVESTIGACIÓN} 
    Como líneas futuras de desarrollo se propone el diseño de una aplicación móvil en la cual se pueda desplegar el sistema creado. Esta propuesta incluye numerosos desafíos, como por ejemplo la optimización del modelo para no ocupar tanto espacio en memoria, la serialización del modelo para embeberlo en un dispositivo, y el manejo de tecnologías de desarrollo de aplicaciones en teléfonos celulares.

AGREGAR EXPRESIVIDAD Expressive TTS, imprimir emociones.

\textcolor{white}{
\parencite{Beranek2005ConcertHA} 
\parencite{busch2005noise}
\parencite{Call2007SoundP}
\parencite{Gardner2002}
\parencite{iso1996-2}
\parencite{kracht2007noise}
\parencite{Lawson2010SoundIntensity}
\parencite{MacLeod2007QuietingWeinberg}
\parencite{Mazer2012Creating}
\parencite{Orellana2006NoiseIT}
\parencite{richardson2009development}
\parencite{Ryherd2008CharacterizingNA}
\parencite{sanz2012tecnicas}
\parencite{Taylor1958NoiseCI}
\parencite{Tsiou2008Noise}
\parencite{text-to-speech-ibm}
\parencite{West2008NoiseIH}}

%-----------------Bibliografia-----------------
\newpage

% \bibliography{bibliografia}
\printbibliography[title={Bibliografía}]

%-----------------ANEXO-----------------
% \newpage
% \appendix
% \section*{ANEXO I. \quad FORMATO INTERNO} 
%     \subsection*{AI 1. \textit{Numeración}}

Las páginas serán enumeradas a partir del Índice de Contenidos, con números romanos colocados en la parte media inferior de cada página. A partir de la Introducción, todas las páginas serán enumeradas con números arábigos ubicados en la parte inferior derecha. No usar la palabra “página” antes de la numeración de las páginas.

\subsection*{AI 2. \textit{Palabras Clave (tamaño 10)}}

Debajo del resumen deberán indicarse hasta cinco (5) palabras clave, también conocidas como descriptores. En la sección del resumen en inglés (abstract) deberán también incluirse las palabras clave (keywords). Se colocarán inmediatamente debajo del resumen (abstract) con la indicación en negrita Palabras Clave (Keywords) y a continuación en la misma línea las palabras. 

\subsection*{AI 3. \textit{Títulos y Subtítulos}}

El título de los capítulos debe numerarse con número arábigo consecutivos, y ser escrito con letra mayúscula y en negrita en tamaño 16 puntos escritos a partir del margen izquierdo en una nueva página. El subtítulo (o si corresponde, el inicio de un texto), comenzará 1 espacio más abajo. El subtítulo será precedido por un número arábigo y se escribirá con letra mayúscula en negrita y tamaño 14 puntos, escrito a partir del margen izquierdo. El subtítulo del 3er. orden (o si corresponde un texto), comenzará 1 espacio más abajo. Los subtítulos de 3er. orden podrán ser idénticos al anterior, pero escritos con letra minúscula (la primera con mayúscula), sin negritas, y el texto comenzará una línea más abajo. Si a continuación del subtítulo de 3er. orden se escribe un subtítulo de 4º orden deberá dejarse un espacio. Los subtítulos de 4º, 5º u orden superior, serán idénticos al de 3er orden y el texto comenzará en la misma línea.

Las palabras y/o frases no deben ser abreviadas en títulos y resumen la primera vez que aparecen. 

\subsection*{AI 4. \textit{Figuras, Tablas, Ecuaciones y Notas al pie (tamaño 10)}}

Las Figuras, Ilustraciones (Estilo Figura) y Tablas deben ser colocadas en secuencia en el texto, y siempre que sea posible próximas de donde son indicadas. Las figuras deben ser de buena calidad (en una impresión deberían notarse todos los detalles que fueran importantes) y llevar una leyenda concisa en su parte inferior. Todas las ilustraciones y figuras deben ser numeradas (Ej “Figura \ref{fig:Tiempo de reverberacion}” etc.) nunca excediendo los márgenes de impresión. En caso de incorporarse a la sección de anexo, la numeración debe estar antepuesta por la letra A y reestablecer la misma a 1 (ej ''Figura \ref{fig:Comparativa R}”).

Las letras en los dibujos o gráficos no deberían ser menores a 1,5 mm de alto. Si es posible, las letras en todas las ilustraciones deben ser del mismo tamaño, y usar el idioma castellano (no inglés). Se recomienda no utilizar solamente colores para identificar características de un gráfico (en caso de mediciones, utilizar símbolos diferentes con cada color, así además se pueden identificar por la forma). Las leyendas de las Tablas (Ej “Tabla 1”) deben ser posicionadas centradas encima de las mismas (Estilo Título Figuras). Tanto para las Figuras como para las Tablas, junto a la identificación debe suministrarse una pequeña descripción explicativa del contenido. En caso de incorporarse a la sección de anexo, la numeración debe estar antepuesta por la letra A y reestablecer la misma a 1 (ej “Tabla \ref{Tab:1}”).

Si la Figura o Tabla es una copia sacada de una de las fuentes bibliográficas, es preferible vectorizar mediante algún programa de dibujo para mejorar la resolución. También debe constar en el título la fuente mediante el número de referencia bibliográfica.
Las ecuaciones y expresiones matemáticas deben estar centradas y numeradas en secuencia, con el número de la ecuación justificado a la derecha y entre paréntesis, utilizando numeración arábiga.
En caso de utilizar notas al pie deben estar referidas con algún símbolo en superíndice (Ej texto†) en el cuerpo del texto y en el pie de la misma página referir el comentario en letra tamaño 10 interlineado simple. Se sugiere que la nota al pie no supere los tres renglones.

Las ecuaciones y expresiones matemáticas deben estar centradas y numeradas en secuencia, con el número de la ecuación a la derecha y entre paréntesis, utilizando numeración arábiga. En ecuaciones de varias líneas, la numeración debe ser ubicada en la última línea. Las fórmulas y el texto deben ser separados por una línea. Las ecuaciones deben ser hechas en la misma fuente del texto, con los índices 3 puntos abajo. Deben usarse símbolos convencionales y unidades SI. Las ecuaciones deben ser citadas en el texto (''Ecuación \eqref{eq: LF}", etc.).

En el texto debe definirse cada símbolo, con sus unidades, y representadas en formato itálica (''donde S es la superficie del elemento en $m^2$, TR es el tiempo de reverberación del recinto a una cierta frecuencia expresado en segundos y $\alpha$ es el coeficiente de absorción sonora, adimensional”).

\begin{equation}
    \label{eq: LF}
    \tau_{d} = \frac{\int_{0}^{\theta_{lim}} \tau \sin \sin \theta \cos \cos \theta \ \dd \theta}{\int_{0}^{\theta_{lim}} \sin \sin \theta \cos \cos \theta \ \dd \theta}
\end{equation}

Para ingresar ecuaciones y se enumeren automáticamente, pueden usar el formato de la ecuación anterior (es una tabla de 1 fila y dos columnas, la primera celda está el editor de ecuaciones y en la segunda celda el número de ecuación): 


    \begin{figure}[h]
        \centering
        \includegraphics[scale=0.9]{Figuras/Tiempo de Reverberacion.png}
        \caption{Comparación de los tiempos de reverberación entre el promedio espacial y la simulación del Hall 4.}
        \label{fig:Tiempo de reverberacion}
    \end{figure}

    \begin{figure}[h]
        \centering
        \includegraphics[scale=0.9]{Figuras/Comparacion reduccion sonora R.png}
        \caption{Comparativa de los índices de reducción sonora R entre la medición de laboratorio, la predicción del modelo propuesto y el programa INSUL para el caso 1.}
        \label{fig:Comparativa R}
    \end{figure}


\begin{table}
\centering
\begin{tblr}{
  cells = {c},
  hline{1-2,7} = {-}{},
}
\textbf{Material}  & \textbf{Espesor \\\ [mm]} & {\textbf{Densidad}\\\textbf{$[Kg/m^3]$}} & {\textbf{Young}\\\textbf{[GPa]}} & \textbf{Poisson} & {\textbf{Factor de perdidas}\\\textbf{interno}} \\
hormigon  & 50,8; 101,6; 140; 160(x2);& 2100 & 30 & 0,2 & 0,03 \\
Vidrio    & 180; 200; 220; 240              & 2500 & 71 & 0,23& 0,02  \\
placa de yeso & 6,4; 9,5; 12,7; 15.9   & 768  & 2 &  0,23& 0,01  \\
laminado  &  3.2; 6.4                       & 1250 & 3 &  0,15& 0,03  \\
HDF       &  50; 70(x4); 100                &  900 &  3.5& 0,2& 0.005
                                       
\end{tblr}
\caption{Lista de materiales utilizados en la comparativa y sus características físicas.}
\label{Tab:1}
\end{table}

\subsection*{AI 5. \textit{Bibliografía}}
Las referencias a la bibliografía utilizada deberán registrarse en el texto entre corchetes con un número arábigo (Ej [5]). La numeración debe ser consecutiva. En caso de citarse más de una referencia se hará separadas por comas dentro del corchete (Ej [5, 18]) y en caso de necesitar la cita de una sucesión consecutiva de referencias se escribirán separadas por una línea (Ej [5-11]).

La numeración correspondiente a las referencias se podrá incluir al final de la tesis (como se especifica más arriba en la descripción de la Estructura) o bien al final de cada capítulo incluyendo solamente las referencias de ese capítulo. Al utilizar las referencias por capítulo pueden quedar referencias repetidas entre capítulos; en este caso cada capítulo debe ser autocontenido con respecto a las referencias.

\end{document}

